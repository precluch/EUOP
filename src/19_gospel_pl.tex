\psalmline{Ewangelia}{}{Łk 24, 13-35}
\psdescr{Poznali Chrystusa przy łamaniu chleba.}{}
\indent\red{\ding{64}} Słowa Ewangelii według świętego Łukasza:

W pierwszy dzień tygodnia dwaj uczniowie Jezusa byli w drodze do wsi, zwanej Emaus, oddalonej sześćdziesiąt stadiów od Jerozolimy. Rozmawiali oni ze sobą o tym wszystkim, co się wydarzyło. Gdy tak rozmawiali i rozprawiali ze sobą, sam Jezus przybliżył się i szedł z nimi. Lecz oczy ich były niejako na uwięzi, tak że Go nie poznali.

On zaś ich zapytał: «Cóż to za rozmowy prowadzicie ze sobą w drodze?» Zatrzymali się smutni. A jeden z nich, imieniem Kleofas, odpowiedział Mu: «Ty jesteś chyba jedynym z przebywających w Jerozolimie, który nie wie, co się tam w tych dniach stało».

Zapytał ich: «Cóż takiego?»

Odpowiedzieli Mu: «To, co się stało z Jezusem z Nazaretu, który był prorokiem potężnym w czynie i słowie wobec Boga i całego ludu; jak arcykapłani i nasi przywódcy wydali Go na śmierć i ukrzyżowali. A myśmy się spodziewali, że właśnie On miał wyzwolić Izraela. Teraz zaś po tym wszystkim dziś już trzeci dzień, jak się to stało. Co więcej, niektóre z naszych kobiet przeraziły nas: Były rano u grobu, a nie znalazłszy Jego ciała, wróciły i opowiedziały, że miały widzenie aniołów, którzy zapewniają, iż On żyje. Poszli niektórzy z naszych do grobu i zastali wszystko tak, jak kobiety opowiadały, ale Jego nie widzieli».

Na to On rzekł do nich: «O nierozumni, jak nieskore są wasze serca do wierzenia we wszystko, co powiedzieli prorocy! Czyż Mesjasz nie miał tego cierpieć, aby wejść do swej chwały?» I zaczynając od Mojżesza poprzez wszystkich proroków wykładał im, co we wszystkich Pismach odnosiło się do Niego.

Tak przybliżyli się do wsi, do której zdążali, a On okazywał, jakoby miał iść dalej. Lecz przymusili Go, mówiąc: «Zostań z nami, gdyż ma się ku wieczorowi i dzień się już nachylił». Wszedł więc, aby zostać z nimi. Gdy zajął z nimi miejsce u stołu, wziął chleb, odmówił błogosławieństwo, połamał go i dawał im. Wtedy otworzyły się im oczy i poznali Go, lecz On zniknął im z oczu. I mówili nawzajem do siebie: «Czy serce nie pałało w nas, kiedy rozmawiał z nami w drodze i Pisma nam wyjaśniał?»

W tej samej godzinie wybrali się i wrócili do Jerozolimy. Tam zastali zebranych Jedenastu i innych z nimi, którzy im oznajmili: «Pan rzeczywiście zmartwychwstał i ukazał się Szymonowi». Oni również opowiadali, co ich spotkało w drodze i jak Go poznali przy łamaniu chleba.

\red{--–}Oto słowo Pańskie.