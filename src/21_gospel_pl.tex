\psalmline{Ewangelia}{}{J 21, 1-14}
\psdescr{Ukazanie się Zmartwychwstałego nad Morzem Tyberiadzkim}{}
\indent\red{\ding{64}} Słowa Ewangelii według świętego Jana:

Jezus znowu ukazał się nad Morzem Tyberiadzkim. A ukazał się w ten sposób:

Byli razem Szymon Piotr, Tomasz, zwany Didymos, Natanael z Kany Galilejskiej, synowie Zebedeusza oraz dwaj inni z Jego uczniów. Szymon Piotr powiedział do nich: «Idę łowić ryby». Odpowiedzieli mu: «Idziemy i my z tobą». Wyszli więc i wsiedli do łodzi, ale tej nocy nic nie złowili.

A gdy ranek zaświtał, Jezus stanął na brzegu. Jednakże uczniowie nie wiedzieli, że to był Jezus.

A Jezus rzekł do nich: «Dzieci, czy nie macie nic do jedzenia?»

Odpowiedzieli Mu: «Nie».

On rzekł do nich: «Zarzućcie sieć po prawej stronie łodzi, a znajdziecie». Zarzucili więc i z powodu mnóstwa ryb nie mogli jej wyciągnąć.

Powiedział więc do Piotra ów uczeń, którego Jezus miłował: «To jest Pan!» Szymon Piotr usłyszawszy, że to jest Pan, przywdział na siebie wierzchnią szatę, był bowiem prawie nagi, i rzucił się w morze. Reszta uczniów dobiła łodzią, ciągnąc za sobą sieć z rybami. Od brzegu bowiem nie było daleko, tylko około dwunastu łokci.

A kiedy zeszli na ląd, ujrzeli żarzące się na ziemi węgle, a na nich ułożoną rybę oraz chleb. Rzekł do nich Jezus: «Przynieście jeszcze ryb, któreście teraz ułowili». Poszedł Szymon Piotr i wyciągnął na brzeg sieć pełną wielkich ryb w liczbie stu pięćdziesięciu trzech. A pomimo tak wielkiej ilości sieć się nie rozerwała. Rzekł do nich Jezus: «Chodźcie, posilcie się!» Żaden z uczniów nie odważył się zadać Mu pytania: «Kto Ty jesteś?», bo wiedzieli, że to jest Pan. A Jezus przyszedł, wziął chleb i podał im, podobnie i rybę.

To już trzeci raz, jak Jezus ukazał się uczniom od chwili, gdy zmartwychwstał.

\red{--–}Oto słowo Pańskie.

