\psalmline{Pierwsze Czytanie}{}{Dz 4, 13-21}
\psdescr{Nie możemy nie mówić tego, co widzieliśmy}{}
\indent Czytanie z Dziejów Apostolskich:

Przełożeni i starsi, i uczeni, widząc odwagę Piotra i Jana, a dowiedziawszy się, że są oni ludźmi nieuczonymi i prostymi, dziwili się. Rozpoznawali w nich też towarzyszy Jezusa. A widząc nadto, że stoi z nimi uzdrowiony człowiek, nie znajdowali odpowiedzi.

Kazali więc im wyjść z sali Rady i naradzali się. Mówili jeden do drugiego: «Co mamy zrobić z tymi ludźmi? Bo dokonali jawnego znaku, oczywistego dla wszystkich mieszkańców Jerozolimy. Przecież temu nie możemy zaprzeczyć. Aby jednak nie rozpowszechniało się to wśród ludu, zabrońmy im surowo przemawiać do kogokolwiek w to imię».

Przywołali ich potem i zakazali w ogóle przemawiać i nauczać w imię Jezusa. Lecz Piotr i Jan odpowiedzieli: «Rozsądźcie, czy słuszne jest w oczach Bożych bardziej słuchać was niż Boga? Bo my nie możemy nie mówić tego, co widzieliśmy i co słyszeliśmy».

Oni zaś ponowili groźby, a nie znajdując żadnej podstawy do wymierzenia im kary, wypuścili ich ze względu na lud, bo wszyscy wielbili Boga z powodu tego, co się stało.

\red{--–}Oto słowo Boże.