\psalmline{Première Lecture}{}{Ac 4, 1-12}
\psdescr{En nul autre que lui, il n’y a de salut.}{}
\indent Lecture du livre des Actes des Apôtres:

En ces jours-là, après la guérison de l’infirme, comme Pierre et Jean parlaient encore au peuple, les prêtres survinrent, avec le commandant du Temple et les sadducéens; ils étaient excédés de les voir enseigner le peuple et annoncer, en la personne de Jésus, la résurrection d’entre les morts. Ils les firent arrêter et placer sous bonne garde jusqu’au lendemain, puisque c’était déjà le soir. Or, beaucoup de ceux qui avaient entendu la Parole devinrent croyants; à ne compter que les hommes, il y en avait environ cinq mille.

Le lendemain se réunirent à Jérusalem les chefs du peuple, les anciens et les scribes. Il y avait là Hanne le grand prêtre, Caïphe, Jean, Alexandre, et tous ceux qui appartenaient aux familles de grands prêtres. Ils firent amener Pierre et Jean au milieu d’eux et les questionnèrent: «Par quelle puissance, par le nom de qui, avez-vous fait cette guérison?»

Alors Pierre, rempli de l’Esprit Saint, leur déclara: «Chefs du peuple et anciens, nous sommes interrogés aujourd’hui pour avoir fait du bien à un infirme, et l’on nous demande comment cet homme a été sauvé. Sachez-le donc, vous tous, ainsi que tout le peuple d’Israël: c’est par le nom de Jésus le Nazaréen, lui que vous avez crucifié mais que Dieu a ressuscité d’entre les morts, c’est par lui que cet homme se trouve là, devant vous, bien portant.

Ce Jésus est la pierre méprisée de vous, les bâtisseurs, mais devenue la pierre d’angle. En nul autre que lui, il n’y a de salut, car, sous le ciel, aucun autre nom n’est donné aux hommes, qui puisse nous sauver.»

\red{--–}Parole du Seigneur.