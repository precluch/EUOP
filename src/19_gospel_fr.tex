\psalmline{Évangile}{}{Lc 24, 13-35}
\psdescr{Il se fit reconnaître par eux à la fraction du pain.}{}
\indent\red{\ding{64}} Évangile de Jésus Christ selon saint Luc:

Le même jour, c’est-à-dire le premier jour de la semaine, deux disciples faisaient route vers un village appelé Emmaüs, à deux heures de marche de Jérusalem, et ils parlaient entre eux de tout ce qui s’était passé. Or, tandis qu’ils s’entretenaient et s’interrogeaient, Jésus lui-même s’approcha, et il marchait avec eux. Mais leurs yeux étaient empêchés de le reconnaître.

Jésus leur dit: «De quoi discutez-vous en marchant?» Alors, ils s’arrêtèrent, tout tristes. L’un des deux, nommé Cléophas, lui répondit: «Tu es bien le seul étranger résidant à Jérusalem qui ignore les événements de ces jours-ci.»

Il leur dit: «Quels événements?»

Ils lui répondirent: «Ce qui est arrivé à Jésus de Nazareth, cet homme qui était un prophète puissant par ses actes et ses paroles devant Dieu et devant tout le peuple: comment les grands prêtres et nos chefs l’ont livré, ils l’ont fait condamner à mort et ils l’ont crucifié. Nous, nous espérions que c’était lui qui allait délivrer Israël. Mais avec tout cela, voici déjà le troisième jour qui passe depuis que c’est arrivé. À vrai dire, des femmes de notre groupe nous ont remplis de stupeur. Quand, dès l’aurore, elles sont allées au tombeau, elles n’ont pas trouvé son corps; elles sont venues nous dire qu’elles avaient même eu une vision: des anges, qui disaient qu’il est vivant. Quelques-uns de nos compagnons sont allés au tombeau, et ils ont trouvé les choses comme les femmes l’avaient dit; mais lui, ils ne l’ont pas vu.»

Il leur dit alors: «Esprits sans intelligence! Comme votre cœur est lent à croire tout ce que les prophètes ont dit! Ne fallait-il pas que le Christ souffrît cela pour entrer dans sa gloire?» Et, partant de Moïse et de tous les Prophètes, il leur interpréta, dans toute l’Écriture, ce qui le concernait.

Quand ils approchèrent du village où ils se rendaient, Jésus fit semblant d’aller plus loin. Mais ils s’efforcèrent de le retenir: «Reste avec nous, car le soir approche et déjà le jour baisse.» Il entra donc pour rester avec eux. Quand il fut à table avec eux, ayant pris le pain, il prononça la bénédiction et, l’ayant rompu, il le leur donna. Alors leurs yeux s’ouvrirent, et ils le reconnurent, mais il disparut à leurs regards. Ils se dirent l’un à l’autre: «Notre cœur n’était-il pas brûlant en nous, tandis qu’il nous parlait sur la route et nous ouvrait les Écritures?»

À l’instant même, ils se levèrent et retournèrent à Jérusalem. Ils y trouvèrent réunis les onze Apôtres et leurs compagnons, qui leur dirent: «Le Seigneur est réellement ressuscité: il est apparu à Simon-Pierre.» À leur tour, ils racontaient ce qui s’était passé sur la route, et comment le Seigneur s’était fait reconnaître par eux à la fraction du pain.

\red{--–}Acclamons la Parole de Dieu.