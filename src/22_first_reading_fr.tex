\psalmline{Première Lecture}{}{Ac 4, 13-21}
\psdescr{Il nous est impossible de nous taire sur ce que nous avons vu et entendu }{}
\indent Lecture du livre des Actes des Apôtres:

En ces jours-là, les chefs du peuple, les Anciens et les scribes constataient l’assurance de Pierre et de Jean et, se rendant compte que c’était des hommes sans culture et de simples particuliers, ils étaient surpris; d’autre part, ils reconnaissaient en eux ceux qui étaient avec Jésus. Mais comme ils voyaient, debout avec eux, l’homme qui avait été guéri, ils ne trouvaient rien à redire.

Après leur avoir ordonné de quitter la salle du Conseil suprême, ils se mirent à discuter entre eux. Ils disaient: «Qu’allons-nous faire de ces gens-là? Il est notoire, en effet, qu’ils ont opéré un miracle; cela fut manifeste pour tous les habitants de Jérusalem, et nous ne pouvons pas le nier. Mais pour en limiter la diffusion dans le peuple, nous allons les menacer afin qu’ils ne parlent plus à personne en ce nom-là.»

Ayant rappelé Pierre et Jean, ils leur interdirent formellement de parler ou d’enseigner au nom de Jésus. Ceux-ci leur répliquèrent: «Est-il juste devant Dieu de vous écouter, plutôt que d’écouter Dieu? À vous de juger. Quant à nous, il nous est impossible de nous taire sur ce que nous avons vu et entendu.»

Après de nouvelles menaces, ils les relâchèrent, faute d’avoir trouvé le moyen de les punir: c’était à cause du peuple, car tout le monde rendait gloire à Dieu pour ce qui était arrivé.

\red{--–}Parole du Seigneur.