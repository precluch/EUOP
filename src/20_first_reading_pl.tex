\psalmline{Pierwsze Czytanie}{}{Dz 3, 11-26}
\psdescr{Bóg naszych ojców wsławił Sługę swego.}{}
\indent Czytanie z Dziejów Apostolskich:

Gdy chromy, uzdrowiony, trzymał się Piotra i Jana, cały lud zdumiony zbiegł się do nich w krużganku, który zwano Salomonowym.

Na ten widok Piotr przemówił do ludu: «Mężowie izraelscy! Dlaczego dziwicie się temu? I dlaczego także patrzycie na nas, jakbyśmy własną mocą lub pobożnością sprawili, że on chodzi? Bóg naszych ojców, Bóg Abrahama, Izaaka i Jakuba wsławił Sługę swego, Jezusa, wy jednak wydaliście Go i zaparliście się Go przed Piłatem, gdy postanowił Go uwolnić. Zaparliście się świętego i sprawiedliwego, a wyprosiliście ułaskawienie dla zabójcy. Zabiliście Dawcę życia, ale Bóg wskrzesił Go z martwych, czego my jesteśmy świadkami.

I przez wiarę w Jego imię temu człowiekowi, którego widzicie i którego znacie, Imię to przywróciło siły. Wiara wzbudzona przez niego dała mu tę pełnię sił, którą wszyscy widzicie.

Lecz teraz wiem, bracia, że działaliście w nieświadomości, tak samo jak przełożeni wasi. A Bóg w ten sposób spełnił to, co zapowiedział przez usta wszystkich proroków, że Jego Mesjasz będzie cierpiał. Pokutujcie więc i nawróćcie się, aby grzechy wasze zostały zgładzone, aby nadeszły od Pana dni ochłody, aby też posłał wam zapowiedzianego Mesjasza, Jezusa, którego niebo musi zatrzymać aż do czasu odnowienia wszystkich rzeczy, co od wieków przepowiedział Bóg przez usta swoich świętych proroków.

Powiedział przecież Mojżesz: “ Proroka jak ja wzbudzi wam Pan, Bóg nasz, spośród braci waszych. Słuchajcie Go we wszystkim, co wam powie. A każdy, kto nie posłucha tego Proroka, zostanie usunięty z ludu ”. Zapowiadali te dni także pozostali prorocy, którzy przemawiali od czasów Samuela i jego następców.

Wy jesteście synami proroków i przymierza, które Bóg zawarł z waszymi ojcami, kiedy rzekł do Abrahama: “ Błogosławione będą w potomstwie twoim wszystkie narody ziemi ”. Dla was w pierwszym rzędzie wskrzesił Bóg Sługę swego i posłał Go, aby błogosławił wam w sprawie odwrócenia się każdego z was od swoich grzechów».

\red{--–}Oto słowo Boże.