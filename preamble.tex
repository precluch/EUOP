\documentclass[11pt,a5paper,openright]{memoir}
\usepackage[
	top=.65in, 
	bottom=.85in, 
	inner=1in, 
	outer=.6in,
	headsep=.15in,
	footskip=.3in,
	nomarginpar,
%	showframe,
	twoside]{geometry}
\usepackage{amsmath}
\usepackage{amsfonts}
\usepackage{amssymb}
\usepackage{graphicx}
\usepackage{polski}
\usepackage{fontspec}
\usepackage{enumitem}
\usepackage{multicol}
\usepackage{gmverse}
\usepackage{lipsum}
\usepackage{pdfpages}
\usepackage{pifont}
\usepackage{fancyhdr}
\usepackage{pgfornament}
\usepackage{afterpage}
\usepackage{polyglossia}
\usepackage{hyperref}

\hypersetup{
	bookmarks=false,
	colorlinks=true,
	pdffitwindow=true,
	pdfstartview={Fit},
	pdfproducer={Dominikanie Kraków},
	pdftitle={EUOP Liturgy Texts},
	pdfauthor={Michał Tarasiuk},
	pdfnewwindow=true,
	linkcolor=black,
	linktoc=page,
	unicode
}

\makeatletter
\renewcommand \dotfill {\leavevmode \cleaders \hb@xt@ .75em{\hss .\hss }\hfill \kern \z@}
\makeatother

\setdefaultlanguage[variant=uk]{english}
\setotherlanguages{french, spanish, polish}

\newcommand\blankpage{%
    \null
    \thispagestyle{empty}%
    \newpage}

\newlength{\myindent}
\setlength{\myindent}{\parindent}

\definecolor{red}{HTML}{ef2d32}%rozowawy
%\definecolor{red}{HTML}{800000}%bordowy

% XeLaTeX
%Adobe Garamond Pro
\fontspec[
	BoldFont={Adobe Garamond Pro},%
	ItalicFont={Adobe Garamond Pro},%
	SmallCapsFont={Adobe Garamond Pro}
	]
{Adobe Garamond Pro}
\setmainfont{Adobe Garamond Pro}[
	Contextuals=Swash,
	Numbers={OldStyle, Proportional},
	Ligatures={Common, Rare}]

\newfontfamily\titling{Eligible}

%%Times New Roman
%\fontspec[
%	BoldFont={Times New Roman},%
%	ItalicFont={Times New Roman},%
%	SmallCapsFont={Times New Roman}]
%{Times New Roman}
%\setmainfont{Times New Roman}[
%	Contextuals=Swash,
%	Numbers={OldStyle, Proportional},
%	Ligatures={Common, Rare}]

%%Sorts Mill Goudy only
%\fontspec[
%	BoldFont={Sorts Mill Goudy},%
%	ItalicFont={Sorts Mill Goudy},%
%	SmallCapsFont={Sorts Mill Goudy}]
%{Sorts Mill Goudy}
%\setmainfont{Sorts Mill Goudy}[
%	Contextuals=Swash,
%	Numbers={OldStyle, Proportional},
%	Ligatures={Common, Rare}]

%% Chapter style
\makechapterstyle{ieop_chpt}{

\setsecheadstyle{\titling\bfseries\Large\centering\uppercase}
\setsubsecheadstyle{\scshape\color{red}}
\setaftersubsecskip{1pt}
\setbeforesubsecskip{1pt}

\renewcommand\printchaptername{}
\renewcommand\printchapternum{}
\renewcommand\chaptitlefont{\titling\bfseries\HUGE\scshape\centering}
\renewcommand\printchaptertitle[1]{\pagestyle{empty}
	\chaptitlefont ##1 \newpage\blankpage}

}
\chapterstyle{ieop_chpt}

\setsecnumdepth{none}
\maxsecnumdepth{none}

\makeatletter
\renewcommand*{\@tocmaketitle}{
	\thispagestyle{plain}
	\begin{center}
		\chaptitlefont Contents
	\end{center}
}
\makeatother

\newcommand\myrule{%
	\pgfornament[color=red, width=0.4\textwidth]{88}}
	
\newcommand\myrulev{%
	\pgfornament[color=red, width=0.4\textwidth, ydelta=-8pt]{88}}

\renewcommand{\headrulewidth}{0pt}

\pagestyle{fancy}
\fancyhf{}
\fancyhead[RE]{\titling\leftmark}
\fancyhead[LO]{\titling\rightmark}
\fancyfoot[CE,CO]{\thepage}
%\fancyfoot[CE,CO]{\myrule \\ \thepage}

\fancypagestyle{plain}{%
  \fancyhf{}%
  \fancyfoot[C]{}%
  \renewcommand{\headrulewidth}{0pt}
  \renewcommand{\footrulewidth}{0pt}
}

\setlength{\parskip}{0em}

\newcommand{\red}[1]{
	{\textcolor{red}{#1}}
}
%\newcommand{\red}[1]{
%	{\textcolor{red}{#1}}
%}

\newcommand{\hourpart}[1]{
	\noindent\textsc{
		\vspace{-1em}
		\red{#1}\\
	}\\
}

\newenvironment{psalm}{
	\setlength{\parskip}{0pt}
	\def\par{\ifvmode\hangindent30pt\noindent\else\endgraf\hangindent40pt\fi}
	\obeylines
	
}{}

\newcommand{\psalmline}[3]{
	\vspace{.3\baselineskip}
	\noindent\makebox[\textwidth][s]{\rlap{{\subsecheadstyle{#1}}}\hfill\red{#2}\hfill\llap{\red{#3}}}
}
%\newcommand{\psalmline}[3]{
%	\vspace{.3\baselineskip}
%	\noindent\makebox[\textwidth][s]{\rlap{{\subsecheadstyle{#1}}}\hfill\red{#2}\hfill\llap{\red{#3}}}
%}

\newcommand{\subsst}[1]{
	{\subsecheadstyle{#1}}
}

\newcommand{\pstitle}[1]{
	\begin{center}
		\red{#1}\\
	\end{center}
}

\newcommand{\psdescr}[2]{
	\noindent\textit{#1}~{#2}\\
}

\newcommand{\psalmdscr}[3]{
	\pstitle{#1}
	\pstitle{#2}
	\psdescr{#3}
	\vspace{+0.3em}
}

\newcommand{\psalmpartt}[6]{
	\psalmline{}{#1}{#2}
	\vspace{-1em}
	\pstitle{#3}
	\psdescr{#4}{#5}
	\vspace{-0.7em}
	\pstitle{#6}
}

\newcommand{\psalmpart}[5]{
	\psalmline{}{#1}{#2}
	\vspace{-1.5em}
	\pstitle{#3}
	\psdescr{#4}{#5}
}

\newcommand{\reading}[3]{
	\noindent\textsc{\red{#1}}
	\hfill
	\red{#2}
	\vspace{0.3em}
	
	{#3}
}

\newcommand{\resp}[6]{
	\indent\red{\small{\textsc{k.}}} {#1}
	\redast {#2}
	\red{\small{\textsc{w.}}} {#3}
	\red{\small{\textsc{k.}}} {#4}
	\red{\small{\textsc{w.}}} {#5}
	\red{\small{\textsc{k.}}} {#6}
	\red{\small{\textsc{w.}}} {#3}
}

\newcommand{\psverse}[1]{
	\hangindent={\parindent + \parindent}
	\noindent{#1}\\
}

\newcommand{\asks}[2]{
	\hangindent=\parindent
	\noindent{#1}
	
	\indent \textit{#2}
	
}

\newcommand{\ask}[2]{
	\begin{samepage}
	\noindent {#1}\\
	\noindent \red{---}{#2}
	
	\end{samepage}
}

\newcommand{\rubric}[1]{
	\indent{\red{#1}}
}
%\newcommand{\rubric}[1]{
%	\indent\red{#1}
%}

\newcommand{\prayer}[1]{
	\begin{center}
		\red{#1}\\
		\vspace{-0.5em}
	\end{center}
}

\newcommand{\reddag}{
		\red{$\dagger$}
}

\newcommand{\redast}{
		\red{*}
}

\newcommand{\lecturedscr}[3]{
	\hspace{-1.15\parindent}\textsc{\red{#1}}\hfill\red{#2}\\	
	\noindent\textit{#3}
}

\newcommand{\lecturecont}[2]{
	\noindent {#1}
	
	{#2}
}

\newcommand{\lecture}[5]{
	\lecturedscr{#1}{#2}{#3}
	
	\lecturecont{#4}{#5}
}

\newcommand{\psalmres}[3]{
	\hspace{-1.15\parindent}\textsc{\red{#1}}\hfill\red{#2}
	
	\red{Ref:} {#3}
}

\newcommand{\psalmverse}[2]{
	\begin{itemize}[topsep=0pt, partopsep=0pt, parsep=0pt, itemsep=0pt, leftmargin=2em]
    	\item[\scriptsize\red{#1}] {#2}
	\end{itemize}
}